% Created 2018-11-05 ma. 10:43
% Intended LaTeX compiler: pdflatex
\documentclass[11pt]{article}
\usepackage[utf8]{inputenc}
\usepackage[T1]{fontenc}
\usepackage{graphicx}
\usepackage{grffile}
\usepackage{longtable}
\usepackage{wrapfig}
\usepackage{rotating}
\usepackage[normalem]{ulem}
\usepackage{amsmath}
\usepackage{textcomp}
\usepackage{amssymb}
\usepackage{capt-of}
\usepackage{hyperref}
\author{Teodor Heggelund}
\date{\today}
\title{}
\hypersetup{
 pdfauthor={Teodor Heggelund},
 pdftitle={},
 pdfkeywords={},
 pdfsubject={},
 pdfcreator={Emacs 25.2.2 (Org mode 9.1.14)}, 
 pdflang={English}}
\begin{document}

\tableofcontents

Cheat sheet: useful hotkeys when working with Jupyter.

NTNU, Konstruksjonsteknikk 2018-10-05.

\section{Start jupyter}
\label{sec:org7b27ea6}
\begin{itemize}
\item Command line option: run \texttt{jupyter lab} from folder where you want notebooks stored.
\item Graphical option: start Jupyter Lab from the start menu or launcher icon
\end{itemize}
\section{Notebook hotkeys}
\label{sec:org20860f3}
I frequently use these shortcuts:

\begin{center}
\begin{tabular}{ll}
\texttt{Esc} & Focus document, stop writing inside code block.\\
\texttt{Enter} & Focus code block\\
\texttt{Up} / \texttt{Down} & Navigate to block above or below\\
\texttt{a} & Insert block above\\
\texttt{b} & Insert block below\\
\texttt{dd} & Delete cell(s)\\
\texttt{Shift+Up} / \texttt{Shift+Down} & Extend selection up/down\\
\texttt{Y} & Change to Code cell (Python)\\
\texttt{M} & Change to Markdown cell (for text, images and formulas)\\
\texttt{Ctrl+Enter} & Evaluate cell\\
\texttt{Shift+Enter} & Evaluate cell and create new cell below\\
\texttt{Ctrl+Shift+C} & View and run any command\\
\end{tabular}
\end{center}

View the command menu to find other shortcuts and commands.
\section{Useful commands}
\label{sec:org34d46e0}
\begin{center}
\begin{tabular}{ll}
\textbf{JupyterLab Reference} & How the whole system works. Opening files, starting notebooks, starting consoles.\\
\textbf{Notebook Reference} & How to use notebooks for running Python code\\
\textbf{Markdown Reference} & How to write text with Markdown\\
\textbf{Run all cells} & Run all cells in the notebook. Useful when opening something you've worked on before.\\
\end{tabular}
\end{center}
\end{document}
